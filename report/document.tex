\documentclass[a4paper, 12pt, brazil]{abntex2}

% \usepackage[utf8]{inputenc} % not needed for XeLaTeX
% \usepackage[brazil]{babel}
\usepackage[useregional]{datetime2}
\usepackage{fullpage} % changes the margin
\usepackage[margin=1in]{geometry}
\usepackage{amsmath,amsthm,amssymb,amsfonts}
% \usepackage[colorinlistoftodos]{todonotes}
% Reajuste de tamanho horizontal de environments
% \usepackage{adjustbox}
% \usepackage{stackrel}
\usepackage{mathtools,bm}
\usepackage{mathrsfs}
\usepackage{float}
\usepackage{enumitem}
% \usepackage{stackrel}
% \usepackage{graphicx}
% \usepackage{dsfont}
% \usepackage{titling}
% \usepackage{emoji}
\usepackage[dvipsnames]{xcolor} % or package color
\usepackage{listings}
\definecolor{background}{rgb}{0.94,0.95,0.96}
\lstdefinestyle{txtOutput}{
    frame=lt,
    %frame=single,
    %fontsize=\small,
    backgroundcolor=\color{background},
    keywordstyle=\ttfamily\color{OliveGreen},
    identifierstyle=\ttfamily\color{CadetBlue}\bfseries, 
    commentstyle=\color{Brown},
    stringstyle=\ttfamily,
    tabsize=1,
    % literate={\ \ }{{\ }}1, reduce the tab width from double to single spacing
    numbers=left,
    xleftmargin=2em,
    breaklines=true,
    breakatwhitespace=true,
    % framexleftmargin=1.5em % Uncomment these two lines if you prefer to have the frame
}

\usepackage{comment}
% Pseudoalgoritmo
\usepackage[portuguese,
            portuguesekw,
            onelanguage,
            linesnumbered,
            ruled,
            lined,
            longend,
            fillcomment
            ]{algorithm2e}
% \usetikzlibrary{automata, positioning, arrows}
% \definecolor{codegray}{rgb}{0.5,0.5,0.5}
% \tikzset{->}
\setlength \parindent{0pt}
\setlength {\marginparwidth}{2cm}

\newcommand{\todaysdate}{\DTMdisplaydate{\the\year}{\the\month}{\the\day}{-1}}

\newcounter{columncounter}[section]
\newcounter{linecounter}[section]

\title{Algoritmos Numéricos\\Trabalho Computacional 2}
\author{Henrique Coutinho Layber}
\date{\todaysdate}

\begin{document}
\vspace{-3cm}
\maketitle
\hrule

%\vspace{5cm}
\section*{Introdução}
A definição do exercício consiste em implementar o método de Euler (o mesmo que Runge-Kutta de 1ª ordem) para resolver sistemas de Equações Diferenciais Ordinárias (EDO) de 1ª ordem, sendo assim possível utilizá-lo em solução de Problemas do Valor Inicial (PVIs) de 2ª ordem. O algoritmo foi implementado em Octave e testado com o problema dado na especificação do trabalho. O código fonte está disponível no repositório do GitHub \footnote{\url{https://github.com/Henriquelay/AN-TC2}}.

\section*{Como o algoritmo funciona}

A obtenção de valores exatos é custosa, portanto usamos métodos rápido para aproximar soluções mais rapidamente, ao depender da aplicação, vale mais a pena um valor rápido e ``preciso o suficiente''. O método de Euler é um dos métodos mais simples para resolver PVIs, porém é um dos mais rápidos, sendo assim, é um método muito útil para se ter em mãos, entretanto, é um método de 1ª ordem, o que significa que ele não é muito preciso, porém, para problemas simples, ele pode ser suficiente. Uma alternativa para o método de Euler é o método de Runge-Kutta de ordem maior, pois trata EDOs de ordem maiores e entretanto são mais precisos que o método de Euler, porém, é mais lentos.

O método de Euler trata-se de subdividir o domínio $[a, b]$ em $m$ fatias e no ponto inicial de cada fatia continuar extrapolando a linha tangente à função no ponto específico, enquanto métodos de ordem maior fazem isso tanto no primeiro ponto quanto em onde o resultado da extrapolação anterior nos levou, ``atualizando'' o valor da tangente em sub-fatias.

\section*{Problemas}
\subsection*{Problema 1}
O problema da validação consiste em uma sistema simples onde temos a solução exata.
$$
\left\{
    \begin{array}{l}
        y_1' = f_1(x, y_1, y_2) = y_2 \\
        y_2' = f_2(x, y_1, y_2) = 2y_1 + y_2 - x^2 \\
        y_1(0) = 1 \\
        y_2(0) = 0
    \end{array}
\right.
$$
Para $m = 5$, obtivemos o seguinte resultado:

$$ 
y_1 = \begin{bmatrix}
    1.0000 \\
    1.0000 \\
    1.0800 \\
    1.2544 \\
    1.5437 \\
    1.9768
\end{bmatrix}, \begin{bmatrix}
    1.0000 \\
    1.0430 \\
    1.1864 \\
    1.4600 \\
    1.9083 \\
    2.5973
\end{bmatrix} = Y_{1_{exact}} 
$$

$$
y_2 = \begin{bmatrix}
    0.0000 \\
    0.4000 \\
    0.8720 \\
    1.4464 \\
    2.1654 \\
    3.0880
\end{bmatrix}, \begin{bmatrix}
    1.0000 \\
    1.0430 \\
    1.1864 \\
    1.4600 \\
    1.9083 \\
    2.5973
\end{bmatrix} = Y_{2_{exact}} 
$$

Gráficos:

\includesvg[scale=0.4]{p1y1} \\
\includesvg[scale=0.4]{p1y2}

Como um teste, troquei o valor de $y_{2}(a)$ para $1.0$, o mesmo que o valor exato, e obtive os resultados:

$$
y_1 = \begin{bmatrix}
    1.0000 \\
    1.2000 \\
    1.5200 \\
    1.9984 \\
    2.6877 \\
    3.6603
\end{bmatrix}, \begin{bmatrix}
    1.0000 \\
    1.0430 \\
    1.1864 \\
    1.4600 \\
    1.9083 \\
    2.5973
\end{bmatrix} = Y_{1_{exact}} 
$$

$$
y_2 = \begin{bmatrix}
    1.0000 \\
    1.6000 \\
    2.3920 \\
    3.4464 \\
    4.8630 \\
    6.7827
\end{bmatrix},\begin{bmatrix}
    1.0000 \\
    1.0430 \\
    1.1864 \\
    1.4600 \\
    1.9083 \\
    2.5973
\end{bmatrix} = Y_{2_{exact}}
$$

Interessantemente, ao definir o valor inicial $y_2(a)$ como o valor exato, esperamos que o erro seja menor, mas o erro aumentou. Isso acontece porque o método de Euler é um método de 1ª ordem, ou seja, ele não é muito preciso, e a folga que erro para um lado (negativo, pois $y_2(0) = 0$ quando $y_{2_{exact}}(0) = 1$) nos deu margem de erro pra extrapolar o declive da curva, fazendo com que o erro aumentasse quando removemos esse ``buffer''. \\
Isso foi acaso, e porque já conhecemos a função exata, mas em um cenário onde não conhecemos a função no ponto, ou até mesmo se conhecemos a função no ponto mas não sua curva, não podemos fazer esse tipo de estimativa propositalmente.

\subsection*{Problema 2}
O problema 2 consiste em resolver o PVI dado na especificação do trabalho, onde não temos a solução exata.
$$
\left\{
    \begin{array}{l}
        y_1' = f_1(x, y_1, y_2) = y_1 + y_2 + 3x \\
        y_2' = f_2(x, y_1, y_2) = 2y_1 - y_2 - x \\
        y_1(0) = 1 \\
        y_2(0) = -1
    \end{array}
\right.
$$
Para $m = 20$, obtivemos o seguinte resultado:

\includesvg[scale=0.4]{p2}

\subsection*{Problema 3}
O problema 3 consiste em resolver um circuito RLC, dado o PVI indicado pelas Leis de Kirchhoff, onde não temos a solução exata. Deve-se obter a corrente $I(t)$ no tempo $t$, onde $t \in [0, 6]$ minutos.
Para $m = 60$, obtivemos o seguinte resultado:

\includesvg[scale=0.4]{p3}

\subsection*{Problema 4}
O problema 4 é um sistema massa-mola, onde não temos a solução exata.
Para $m = 400$, obtivemos o seguinte resultado:

\includesvg[scale=0.4]{p4}

\subsection*{Problema 5}
O problema 5 é um sistema de presa-predador , onde não temos a solução exata.
Para $m = 240$, obtivemos o seguinte resultado:

\includesvg[scale=0.4]{p5}


\end{document}
