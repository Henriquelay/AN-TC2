\section*{Problemas}
\subsection*{Problema 1}
O problema da validação consiste em uma sistema simples onde temos a solução exata.
$$
\left\{
    \begin{array}{l}
        y_1' = f_1(x, y_1, y_2) = y_2 \\
        y_2' = f_2(x, y_1, y_2) = 2y_1 + y_2 - x^2 \\
        y_1(0) = 1 \\
        y_2(0) = 0
    \end{array}
\right.
$$
Para $m = 5$, obtivemos o seguinte resultado:

$$ 
y_1 = \begin{bmatrix}
    1.0000 \\
    1.0000 \\
    1.0800 \\
    1.2544 \\
    1.5437 \\
    1.9768
\end{bmatrix}, \begin{bmatrix}
    1.0000 \\
    1.0430 \\
    1.1864 \\
    1.4600 \\
    1.9083 \\
    2.5973
\end{bmatrix} = Y_{1_{exact}} 
$$

$$
y_2 = \begin{bmatrix}
    0.0000 \\
    0.4000 \\
    0.8720 \\
    1.4464 \\
    2.1654 \\
    3.0880
\end{bmatrix}, \begin{bmatrix}
    1.0000 \\
    1.0430 \\
    1.1864 \\
    1.4600 \\
    1.9083 \\
    2.5973
\end{bmatrix} = Y_{2_{exact}} 
$$

Gráficos:

\includesvg[scale=0.4]{p1y1} \\
\includesvg[scale=0.4]{p1y2}

Como um teste, troquei o valor de $y_{2}(a)$ para $1.0$, o mesmo que o valor exato, e obtive os resultados:

$$
y_1 = \begin{bmatrix}
    1.0000 \\
    1.2000 \\
    1.5200 \\
    1.9984 \\
    2.6877 \\
    3.6603
\end{bmatrix}, \begin{bmatrix}
    1.0000 \\
    1.0430 \\
    1.1864 \\
    1.4600 \\
    1.9083 \\
    2.5973
\end{bmatrix} = Y_{1_{exact}} 
$$

$$
y_2 = \begin{bmatrix}
    1.0000 \\
    1.6000 \\
    2.3920 \\
    3.4464 \\
    4.8630 \\
    6.7827
\end{bmatrix},\begin{bmatrix}
    1.0000 \\
    1.0430 \\
    1.1864 \\
    1.4600 \\
    1.9083 \\
    2.5973
\end{bmatrix} = Y_{2_{exact}}
$$

Interessantemente, ao definir o valor inicial $y_2(a)$ como o valor exato, esperamos que o erro seja menor, mas o erro aumentou. Isso acontece porque o método de Euler é um método de 1ª ordem, ou seja, ele não é muito preciso, e a folga que erro para um lado (negativo, pois $y_2(0) = 0$ quando $y_{2_{exact}}(0) = 1$) nos deu margem de erro pra extrapolar o declive da curva, fazendo com que o erro aumentasse quando removemos esse ``buffer''. \\
Isso foi acaso, e porque já conhecemos a função exata, mas em um cenário onde não conhecemos a função no ponto, ou até mesmo se conhecemos a função no ponto mas não sua curva, não podemos fazer esse tipo de estimativa propositalmente.

\subsection*{Problema 2}
O problema 2 consiste em resolver o PVI dado na especificação do trabalho, onde não temos a solução exata.
$$
\left\{
    \begin{array}{l}
        y_1' = f_1(x, y_1, y_2) = y_1 + y_2 + 3x \\
        y_2' = f_2(x, y_1, y_2) = 2y_1 - y_2 - x \\
        y_1(0) = 1 \\
        y_2(0) = -1
    \end{array}
\right.
$$
Para $m = 20$, obtivemos o seguinte resultado:

\includesvg[scale=0.4]{p2}

\subsection*{Problema 3}
O problema 3 consiste em resolver um circuito RLC, dado o PVI indicado pelas Leis de Kirchhoff, onde não temos a solução exata. Deve-se obter a corrente $I(t)$ no tempo $t$, onde $t \in [0, 6]$ minutos.
Para $m = 60$, obtivemos o seguinte resultado:

\includesvg[scale=0.4]{p3}

\subsection*{Problema 4}
O problema 4 é um sistema massa-mola, onde não temos a solução exata.
Para $m = 400$, obtivemos o seguinte resultado:

\includesvg[scale=0.4]{p4}

\subsection*{Problema 5}
O problema 5 é um sistema de presa-predador , onde não temos a solução exata.
Para $m = 240$, obtivemos o seguinte resultado:

\includesvg[scale=0.4]{p5}
